\vspace{-2ex}
\section{Related Work}

Throughout the years, there have been increasing efforts towards code-size
optimizations.
%Most of the classic optimizations try to find semantically equivalent programs
%that require fewer instructions to execute, such as dead code elimination,
%common subexpression elimination, and others~\cite{cocke70,massalin87,knoop94}.
Compiler optimizations for code-size reduction exist since the very begining
of optimizing compilers.
These optimizations can be divided in two main categories:
those that replace a piece of code by a smaller but semantically equivalent code,
changing the instructions and operations performed~\cite{massalin87,tanenbaum82};
and those that remove or combine redundant code~\cite{cocke70,knoop94,ernst97,
                                              cooper99,debray00,chen03,loki04}.
One optimization from this second category is function
merging.

Google developed an optimization for the \textit{gold} linker that merges
identical functions on a bit-level~\cite{tallam10,kwan12}.
After placing each function in a separate ELF section, they identify function
sections that have their \textit{text} section bit-identical and also their
relocations point to identical sections.
%A simpler version of this optimization was also offered by the MSVC linker~\cite{msvc-icf};
A similar optimization for merging identical functions %, but implemented at the IR level,
is also offered by both GCC and LLVM~\cite{llvm-fm,livska14}.
In Section~\ref{sec:background}, we presented a detailed description of the
related work on function-merging, including the state-of-the-art~\cite{edler14}.


%\subsection{Code Factoring}

%Function merging and code factoring are different techniques for solving the
%same fundamental problem of duplicated code.
Code factoring is a different technique for solving the same fundamental problem
of duplicated code.
%While the former works by merging similar functions, the latter works by
%factoring out duplicated code~\cite{loki04}.
%Instead of merging similar functions, code factoring works by factoring out
%duplicated code into separate functions~\cite{loki04}.
Code factoring can be applied at different levels of the program~\cite{loki04}.
Local factoring, also known as local code motion, moves identical instructions
from multiple basic blocks to either their common predecessor or successor,
whenever valid~\cite{knoop94,briggs94,loki04}.
Procedural abstraction (or outlining) finds identical code
that can be extracted into a separate function, replacing all replicated
occurrences with a function call~\cite{loki04,dreweke07}.

Procedural abstraction differs from function merging as it usually works on
single basic blocks or single-entry single-exit regions.
Moreover, it only works for identical segments of code, and every identical
segment of code is extracted into a separately new function.
Function merging, on the other hand, works on whole functions, which can be
identical or just partially similar, producing a single merged function.

However, all these techniques are orthogonal to the proposed optimization and
could complement each other at different stages of the compilation pipeline.
