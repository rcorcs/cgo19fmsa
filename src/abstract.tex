\begin{abstract}
Resource-constrained devices for embedded systems are becoming increasingly important.
In such systems, memory is highly restrictive, making code size in most cases
even more important than performance.
Compared to more traditional platforms, memory is a larger part of the cost and
code occupies much of it. Despite that, compilers make little effort to reduce
code size.
One key technique attempts to merge the bodies of similar functions.
However, production compilers only apply this optimization to identical functions,
while research compilers improve on that by merging the few functions with
identical control-flow graphs and signatures.
Overall, existing solutions are insufficient and we end up having to either increase cost by adding more memory or remove functionalities from programs.

We introduce a novel technique that can merge arbitrary functions through sequence
alignment, a bioinformatics algorithm for identifying regions of similarity
between sequences. We combine this technique with an intelligent exploration
mechanism to direct the search towards the most promising function pairs. Our
approach is more than $3\times$ better than the state-of-the-art, reducing code
size by up to 22\%, with an overall average of 5.3\%, while introducing an
average compilation-time overhead of only 20\%. When aided by profiling information,
this optimization can be deployed without any significant impact
on the performance of the generated code.

\end{abstract}

%In recent years, resource-constrained devices are becoming increasingly important.
%At the same time, some programs for these devices can have binaries of several megabytes in size, with ever-increasing complexity.
%Savings in code size enables more features to be included.
%However, valuable resources are still wasted as a result of unnecessarily large
%binaries caused by the weakness of current code-size optimizations.
%Most industrial-strength compilers offer an important code-size optimization
%that reduces redundant code by merging similar functions.
%In this paper, we propose a novel technique for merging functions
%that addresses major limitations of the state-of-the-art
%with a fundamentally different approach. We embed this technique in
%a ranking-based exploration mechanism so that we can focus
%the optimization on promising pairs of functions.
%Our approach is more than $3\times$ better than the state-of-the-art,
%reducing code size of programs by up to 22\%, with an overall average of 5.6\%,
%while introducing an average compilation-time overhead of only 20\%.
%Moreover, aided by profiling information, this optimization can be deployed
%without any significant impact on the performance of the generated code.
