\begin{abstract}
%Resource-constrained devices are becoming increasingly important.
%However, valuable resources are still wasted as a result of unnecessarily large 
%binaries caused by the weakness of current code-size optimizations.
In recent years, resource-constrained devices are becoming increasingly important.
At the same time, some programs for these devices can have binaries of several megabytes in size, with ever-increasing complexity.
Savings in code size enables more features to be included.
However, valuable resources are still wasted as a result of unnecessarily large 
binaries caused by the weakness of current code-size optimizations.
Most industrial-strength compilers offer an important code-size optimization
that reduces redundant code by merging similar functions.
In this paper, we propose a novel technique for merging functions
that addresses major limitations of the state-of-the-art
with a fundamentally different approach. We embed this technique in
a ranking-based exploration mechanism so that we can focus
the optimization on promising pairs of functions.
Our approach is more than $3\times$ better than the state-of-the-art,
reducing code size of programs by up to 22\%, with an overall average of 5.6\%,
while introducing an average compilation-time overhead of only 20\%.
Moreover, aided by profiling information, this optimization can be deployed
without any significant impact on the performance of the generated code.

\end{abstract}

