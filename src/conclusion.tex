\vspace{-1ex}
\section{Conclusion}

We have presented a novel technique, based on sequence alignment, for merging
functions that addresses major limitations of the existing solutions.
We have also propose a ranking-based exploration mechanism so that we can focus
the optimization on promising pairs of functions, reducing considerably the
compilation-time overhead, compared to the oracle's quadratic exploration.
With this framework, our prototype introduces an average
compilation-time overhead of only 20\%.
Our optimization shows code-size reductions up to 22\%, with an overall average
of about 5.6\%.
This optimization can be carried on without any significant impact on
performance when profiling information is used.

%Allowed instruction reordering could be performed in order to improve coverage
%of the proposed function merging strategy.
%Although we perform sequence alignment directly on the functions after
%linearization, some functions can have their instructions reodered without
%changing their semantics.
%This instruction reodering could potentially improve the alignment of two given
%functions.
%However, it is a constly operation that can significantly degrade compilation
%time.

For future work, we plan to focus on improving the ranking mechanism to reduce
compilation time even further.
%A better compilation time can be achieved by improving the ranking heuristic
%and also by integrating the function merging to a summary-based parallel
%link-time optimization framework, such as ThinLTO in LLVM.
We also plan to work on the linearization of the candidate functions, allowing
instruction reordering to maximize the number of matches between the functions.
%Have a linearization of the candidates based on the current function:
%Lin(F1) = canonical linearization of F1, then linearize F2 based on Lin(F1).

